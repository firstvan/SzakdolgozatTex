\documentclass[a4paper, 12pt, oneside]{book}
\usepackage[utf8]{inputenc}
\usepackage{geometry}
\geometry{lmargin=3cm, rmargin=2cm, tmargin=3cm, bmargin=3cm}
\usepackage[magyar]{babel}
\usepackage{import}
\usepackage{t1enc}
\usepackage{minted}
\usepackage{times}
\usepackage{indentfirst}

\linespread{1.25}
\setlength{\parskip}{1em}

\usepackage{graphicx}
\graphicspath{ {images/} }

\title{SZAKDOLGOZAT}
\author{Istvan Orban}
\date{March 2016}

\begin{document}


\begin{titlepage}
	\centering
	\vspace*{4cm}
	{\scshape\LARGE SZAKDOLGOZAT\par}
	\vfill
	{\bfseries \raggedleft Orbán István Sándor\par}
	\vfill

% Bottom of the page
	{\LARGE Debrecen \\ 2016 \par}
	
    \newpage
    \pagestyle{empty}
	\centering
	{\scshape \large Debreceni Egyetem \\ Informatikai Kar\par}
	\vfill
	{\LARGE Webes alkalmazás fejlesztése Scala nyelven\par}
	\vfill
	
    {\textbf{Témavezető:}
    \hfill \hfill \textbf{Készítette:} \\
    Dr. Szathmáry László
    \hfill \hfill 
    Orbán István Sándor \\
    egyetemi docens
    \hfill \hfill Programtervező Informatikus\par}
    
    \vspace{2.5cm}
	{\large Debrecen \\ 2016 \par}
\end{titlepage}


\tableofcontents


\chapter{Bevezetés}
\import{Chapter_01/}{Motivation.tex}
\import{Chapter_01/}{Purpose.tex}

\chapter{Felhasznált eszközök és technológiák}
\import{Chapter_02/}{Backend.tex}
\import{Chapter_02/}{Frontend.tex}
\import{Chapter_02/}{Database.tex}

\chapter{Az alkalmazás részletes ismertetése}
\import{Chapter_03/}{Intro.tex}
\import{Chapter_03/}{GUI.tex}
\import{Chapter_03/}{Construct.tex}

\chapter{Összefoglalás}
\import{Chapter_04/}{Summary.tex}


\chapter*{Köszönetnyílvánítás}
\addcontentsline{toc}{chapter}{Köszönetnyílvánítás}
\title{Köszönetnyílvánítás}
...

\newpage
\addcontentsline{toc}{chapter}{Irodalomjegyzék}
\begin{thebibliography}{11}
\bibitem{Scala}
http://www.scala-lang.org/what-is-scala.html
\bibitem{Play}
https://www.playframework.com/documentation/2.4.x/Home
\bibitem{JavaScript}
http://javascript.about.com/od/reference/p/javascript.htm
\bibitem{JavaScript2}
Jon Duckett: Beginning HTML, XHTML, CSS, and Java
\bibitem{MongoDB 1}
http://searchdatamanagement.techtarget.com/definition/MongoDB
\bibitem{MongoDB 2}
Dr. Szathmáry László: A MongoDB adatbázis-kezelő rendszer
\bibitem{HTML}
http://www.tutorialspoint.com/html/

\end{thebibliography}
\end{document}
