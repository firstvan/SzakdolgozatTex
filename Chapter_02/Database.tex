\section{Adatbázis}
\subsection{MongoDB}
A MongoDB egy nem-relációs adatbázis, melynek adatmodellje dokumentum-orientált. Egyike az úgynevezett NoSQL típusú adatbázisoknak. Ennek lényege, hogy az adatokat nem táblákban és rekordokba helyezi el, hanem használja a JSON-os struktúrát. Ez a struktúra dokumentumokból épül fel, amelybe kulcs-érték párok vannak. A másik alapegysége a dokumentumok egy kollekciója.\\
A MongoDB első változata 2009-ben, tehát aránylag új.

Főbb jellemzői a hagyományos adatbázis struktúrákkal szemben:
\begin{itemize}
  \item nem támogatja a join műveleteket
  \item nincsennek tranzakciók
  \item a sémák rugalmasak, így bármikor bővíthető új tipusokkal egy kollekció
\end{itemize}

A dokumentumtárolás és adatcsere formája a BSON, amely egy bináris JSON formátum.

\subsection{Casbah}
A Casbah egy eszköztár a Scala nyelvhez MongoDB adatbázis eléréséhez és kezeléséhez. Ez teremt kapcsolatot az adatbázis és a program között. Célja, hogy a  Scala nyelvi konvencióknak megfelelve könnyű kezelést biztosítson MongoDB adatbázisok kezelésére. Úgy épült fel, hogy kihasználja a Scala nyelv előnyeit, illetve támogassa a tipuskonverziókat az adatbázis és a programnyelv típusai között. Szinte minden natív parancshoz kínál egy neki megfelelő függvényt, amellyel könnyen elvégézhető a kívánt feladat.