\section{Áttekintés}
Az alkalmazás egy cég területi képviselőinek munkáját hivatott megkönnyíteni. A területi képviselő az a személy, aki egy cég nevében járja a boltokat és az igényeknek megfelelő rendelést ad le a cégének. Ez a webes alkalmazás ezt hivatott megkönnyíteni. Szerkezetét tekintve hasonlít egy webshopra, viszont vannak eltérések. Egyik fő eltérés, hogy a felhasználók nem a vásárlók, hanem a közvetítő személy. Emiatt az alkalmazás várható felhasználói száma, mindenféleképpen kevésnek mondható, 2-50 emberig. Ellenben a vásárlói szám már megfelel egy átlagos webshopnak a felhasználói számával. Továbbá még, ami eltérés, hogy itt inkább a gyorsaságra és a praktikumra törekszünk, a vizuális élmények helyett. Azért is emelem ki a gyorsaságot, hiszen egy területi képviselő több megyének a boltjait is látogatja, így egy mobil eszköz segítségével használja az alkalmazást, és mint tudjuk, a mobil internet gyorsasága helyenként eltérő lehet, viszont minden körülmény között működnie kell a szoftvernek.

Ezen alkalmazás hasonló tulajdonsága a webshopokkal, hogy itt is termékeket prezentálunk, de míg a webshopokban jelentős mennyiségű termékek közül választhatunk, addig ebben a szoftverben csak az adott cég által kínált termékek vannak jelen. Ennek ellenére az alkalmazás úgy lett megírva, hogy ad egy általános felületet és az alatta lévő adatbázis, azaz a termékek könnyen lecserélhetőek, bővíthetőek. Így elmondható, hogy ez az alkalmazás egy elég szűk keresztmetszetű réteg munkáját segíti, viszont bármely termékeket áruló cég alkalmazni tudja. Az én alkalmazásom kozmetikai és vegyi árukat tartalmazó adatbázist használ, és szerkezete ennek megfelelően lett kialakítva. 

Mind a felület, mind az alkalmazás úgy lett kialakítva, hogy minden kézre álljon, és minden a lehető legkönnyebben elérhető legyen. Az alkalmazás természetesen a modell-nézet-vezérlő (MVC vagyis model-view-controller) szerkezeti minta alapján lett kialakítva, amely a különböző rétegek elkülönítésére szolgál. Különböző rétegek alatt a felhasználói felületet, üzleti logikát, és az adatok kezelését, adatbázis elérést értem. Ez a minta elengedhetetlen ahhoz, hogy az adatbázis és a felhasználói felület könnyen megváltoztatható legyen.

Munkám során törekedtem egy sokoldalú, könnyen személyre szabható, és komfortosan kezelhető alkalmazás fejlesztésére. 
